\documentclass[alternative,compact,blue]{/Users/tyler/Git/thchang-style/styles/lualatex/yaac-another-awesome-cv}
\usepackage{csquotes}
\usepackage{etaremune}

% Set URL style
\renewcommand{\UrlFont}{\small\rm}
\sloppy

% Define bullitem macro for adding bullet-points
\def\bullitem{\parskip 1pt \par\hangindent=15pt \hangafter=1
\noindent\hbox to 20pt{\hfil$\bullet$\hfil}\ignorespaces}

% Define tab macros
\newcommand\tabboxsmall[1]{\noindent\hbox to 0.07\textwidth{#1\hfill}\hskip 2em}
\newcommand\tabboxmed[1]{\noindent\hbox to 0.18\textwidth{#1\hfill}\hskip 2em}
\newcommand\tabboxlarge[1]{\noindent\hbox to 0.47\textwidth{#1\hfill}\hskip 2em}


\begin{document}

\name{{\bf Tyler H. Chang}}{ -- CV}
\tagline{Building Open Source Software for Machine Learning and Optimization at
Argonne National Laboratory}
\socialinfo{
	\email{thchang@vt.edu}
    \smartphone{+1 (443) 417-7546}
	\address{5674 Walnut Ave, Unit 1C, Downers Grove, IL 60516}\\
	\website{https://thchang.github.io}{https://thchang.github.io}
	\github{https://github.com/thchang}
}

\makecvheader

\makecvfooter
	\today
	{T.~H.~Chang - CV}
	{\thepage}

\goodbreak

\sectionTitle{Summary and Goals}{\faBookmark}

\nopagebreak \bigskip \nopagebreak

I am {\bf passionate about building}
robust and scalable software systems
to solve complex real-world problems by leveraging state-of-the-art
machine learning methods.
I have nearly 8 years of experience building software systems
for diverse scientific applications.
In that time,
I have {\bf led 4 open source projects} and {\bf contributed to 3 others}.

Now I am looking for an opportunity to build and scale similar systems in an
industrial setting.

\nopagebreak \bigskip

\sectionTitle{Recent Work Experience}{\faBookmark}

\nopagebreak \bigskip \nopagebreak

\tabboxmed{Jun 2020 - Present.} {\bf Postdoctoral appointee: Argonne National Laboratory}, Math and Computer Science Division
\bullitem Built production-grade open source software for leveraging machine learning in numerical optimization workflows
\bullitem Explored trade-offs between accuracy and latency in neural network architecture search on 1000+ node HPCs
\bullitem Reduced time and cost of material manufacturing by factor of over 100x via active learning in a wet-lab environment

\medskip

\tabboxmed{Aug 2016 - May 2020.} {\bf Research fellow: Virginia Tech}, Dept. of Computer Science
\bullitem Researched and implemented novel methods for scientific machine learning and numerical optimization
\bullitem Designed parallel algorithms and software for error-bounded machine learning and blackbox optimization
\bullitem Achieved 3x reduction in performance variability in leadership-class HPC at Argonne via above techniques

\medskip

\tabboxmed{Feb 2016 - Aug 2016.} {\bf Research assistant: Old Dominion University}, Dept. of Computer Science
\bullitem Aided in parallelizing NASA's FUN3D CFD kernel on NVIDIA GPUs using CUDA and MPI

\goodbreak

\bigskip

\sectionTitle{Education}{\faBookmark}

\nopagebreak \bigskip \nopagebreak

Ph.D., May 2020, Computer Science, Virginia Polytechnic Institute \& State University (Virginia Tech)

\smallskip

B.S., May 2016, Computer Science \& Mathematics (double-major), Virginia Wesleyan University, {\sl summa cum laude}

\goodbreak

\bigskip

\sectionTitle{Technical Skills}{\faBookmark}

\nopagebreak \bigskip \nopagebreak

\hangafter=1 \hangindent=0.22\textwidth 
\tabboxmed{\bf Mathematical Skills:}numerical optimization, scientific machine learning, approximation theory, computational geometry

\hangafter=1 \hangindent=0.22\textwidth 
\tabboxmed{\bf Computing Skills:}high-performance computing, open source software design, data structures \& algorithms

\hangafter=1 \hangindent=0.22\textwidth 
\tabboxmed{\bf Languages (expert):}Python, Fortran

\hangafter=1 \hangindent=0.22\textwidth 
\tabboxmed{\bf Libraries (expert):}BLAS, jax, LAPACK, numpy, OpenMP, scipy

\hangafter=1 \hangindent=0.22\textwidth 
\tabboxmed{\bf Languages (proficient):}C, C++, Java, Matlab

\hangafter=1 \hangindent=0.22\textwidth 
\tabboxmed{\bf Libraries (proficient):}CUDA, keras, matplotlib, MPI, pandas, plotly/dash, pytorch, scikit-learn

\hangafter=1 \hangindent=0.22\textwidth 
\tabboxmed{\bf Tools/Workflow:}CI/CD, GitFlow, GitHub Actions, kernprof, perf, pytest, qsub, slurm, sphinx

\goodbreak

\bigskip

\sectionTitle{Publicly Available Software (as Creator or Co-Creator)}{\faBookmark}

\nopagebreak \bigskip \nopagebreak

\begin{etaremune}
\item 2022 - Present. {ParMOO}: Machine learning surrogate-assisted simulation optimization on HPCs. Release: 0.3.1\\
Devs: {\bf T. H. Chang} (lead), S. M. Wild, and H. Dickinson \hskip 2em In {\tt Python 3}\\
{\tt git:} \url{https://github.com/parmoo/parmoo} $\quad$ {\bf 68 stars} $\quad$
Used by: Argonne, Meta, Imperial College London, and more
\item 2017 - Present. {DelaunaySparse}: Interpretable machine learning via Delaunay interpolation.\\
Devs: {\bf T. H. Chang} (lead), T. C. H. Lux, and L. T. Watson \hskip 2em In {\tt Fortran 2003} with {\tt C}, {\tt Python}, and CL interfaces\\
{\tt git:} \url{https://github.com/vtopt/DelaunaySparse} $\quad$ {\bf 17 stars} $\quad$
Used by:  Argonne, LLNL, DC Children's Hospital, and more\\
{\bf currently in discussion for inclusion in scipy ``interpolate'' module}
\item 2020 - 2022. {VTMOP}: Parallel solver for computationally expensive multiobjective optimization problems. \\
Devs: {\bf T. H. Chang} (lead) and L. T. Watson \hskip 2em In {\tt Fortran 2008} with {\tt Python} interface\\
{\tt git:} \url{https://github.com/vtopt/VTMOP}
\item 2019 - 2020. {QAML}: Library for converting Python code into quantum annealing circuits.\\
Devs: T. C. H. Lux (lead), {\bf T. H. Chang}, and S. S. Tipirneni \hskip 2em In {\tt Python 3}\\
{\tt git:} \url{https://github.com/tchlux/qaml}
\end{etaremune}

\goodbreak

\bigskip

\sectionTitle{Selected Publications (from 32 indexed on Scholar)}{\faBookmark}

\nopagebreak \bigskip \nopagebreak

\begin{etaremune}
\item 2023. {\bf T. H. Chang}, J. R. Elias, S. M. Wild, S. Chaudhuri, and J. A. Libera. A framework for fully autonomous design of materials via multiobjective optimization and active learning: challenges and next steps. {\sl In 11th Intl. Conf. on Learning Representation (ICLR 2023), Workshop on Machine Learning for Materials (ML4Materials)}. {\tt url:} \url{https://openreview.net/forum?id=8KJS7RPjMqG}
\item 2023. {\bf T. H. Chang} and S. M. Wild. ParMOO: a Python library for parallel multiobjective simulation optimization. {\sl Journal of Open Source Software} 8(82), Article 4468, 5 pages. {\tt doi:} \href{https://doi.org/10.21105/joss.04468}{10.21105/joss.04468}
\item 2022. {\bf T. H. Chang}, L. T. Watson, J. Larson, N. Neveu, W. I. Thacker, S. Deshpande, and T. C. H. Lux. Algorithm 1028: VTMOP: Solver for blackbox multiobjective optimization problems. {\sl ACM Transactions on Mathematical Software} 48(3), Article 36, 34 pages. {\tt doi:} \href{https://doi.org/10.1145/3529258}{10.1145/3529258}
\item 2020. {\bf T. H. Chang}, L. T. Watson, T. C. H. Lux, A. R. Butt, K. W. Cameron, and Y. Hong. Algorithm 1012: DELAUNAYSPARSE: Interpolation via a sparse subset of the Delaunay triangulation in medium to high dimensions. {\sl ACM Transactions on Mathematical Software} 46(4), Article 38, 20 pages. {\tt doi:} \href{https://doi.org/10.1145/3422818}{10.1145/3422818}
\end{etaremune}

\goodbreak

\bigskip

\sectionTitle{Notable Contributions to Publicly Available Software}{\faBookmark}

\nopagebreak \bigskip \nopagebreak

\begin{etaremune}
\item 2023 - Present. DeepHyper: Scalable asynchronous neural architecture \& hyperparameter search for deep learning\\
{\bf My Contributions}: DeepHyper team member adding multiobjective search features and benchmark problems\\
{\tt git:} \url{https://github.com/deephyper/deephyper} $\quad$ {\tt Python 3} $\quad$ {\bf 254 stars}
\item 2019 - Present. libEnsemble: Python toolkit for coordinating asynchronous dynamic ensmbles of calculations\\
{\bf My Contributions}: Adding new ``generator'' techniques and providing example use-cases\\
{\tt git:} \url{https://github.com/Libensemble/libensemble} $\quad$ {\tt Python 3} $\quad$ {\bf 59 stars}
\item 2016. Fun3D: Fully unstructured Navier-Stokes (by NASA Langley)\\
{\bf My Contributions}: Optimizing block-sparse linear system solver for parallelization on NVIDIA GPUs\\
{\tt web:} \url{https://fun3d.larc.nasa.gov} $\quad$ {\tt Fortran 90}
\end{etaremune}

\goodbreak

\bigskip

\sectionTitle{Funding and Awards}{\faBookmark}

\nopagebreak \bigskip \nopagebreak

\textbf{\large Research Funding Raised}

\nopagebreak \medskip \nopagebreak

\begin{etaremune}
\item Mar 2024 - Present. {\bf Key contributor (multiobjective search thrust lead)}, \$400K/y for 1 year. {\sl High performance computing for development of critical thermodynamic inputs for next generation thermal barrier coatings}, external grant (HPC for Manufacturing, DE-AC02-05CH11231)
\item Mar 2023 - Sep 2023. {\bf Co-PI (design optimization thrust lead)}, \$50K/y for 1 year. {\sl A Scalable Multi-Physics Optimization Framework for Particle Accelerator Design}, institutional seed funding (LDRD 2023-0246)
\item Jun 2019 - Dec 2019. {\bf Primary awardee}, \$3K/mo for 6 months. {\sl An Adaptive Weighting Scheme for Multiobjective Optimization}, DOE award for PhD students (DE-SC0014664)
\end{etaremune}

\goodbreak

\medskip

\textbf{\large Awards and Accomplishments}

\nopagebreak \medskip \nopagebreak

\begin{etaremune}
\item Jan 2021. Nominee for Outstanding Dissertation Award, Virginia Tech, Graduate School
\item Apr 2016. Outstanding Student in Computer Science \& Mathematics, Virginia Wesleyan University
\item Feb 2016. ACM International Collegiate Programming Competition (ICPC), winning team for CNU site, VA, USA
\item Feb 2015. ACM International Collegiate Programming Competition (ICPC), winning team for CNU site, VA, USA
\end{etaremune}

\goodbreak \bigskip

\sectionTitle{Leadership and Service}{\faBookmark}

\nopagebreak \bigskip \nopagebreak

\textbf{\large Interns Advised}

\nopagebreak \medskip \nopagebreak

\tabboxmed{ Jun 2022 - Aug 2022.} Manisha Garg (UIUC), NSF MSGI (PhD student intern) at Argonne

\smallskip

\tabboxmed{ Jun 2022 - Aug 2022.} Hyrum Dickinson (UIUC), DOE SULI (undergraduate intern) at Argonne

\goodbreak

\medskip

\textbf{\large Teaching}

\nopagebreak \medskip \nopagebreak

\tabboxmed{Jan 2022 - Present.}{\bf Adjunct Professor: College of DuPage}, Dept. of Computer and Info. Science (Intro to Python)

\smallskip

\tabboxmed{Jan 2020 - May 2020.}{\bf Instructor of Record: Virginia Tech}, Dept. of Computer Science (Data structures and algorithms)

\goodbreak

\medskip

\textbf{\large Journal / Conference Referee}

\nopagebreak

\medskip

\nopagebreak

INFORMS Journal on Computing (2023--Present); ACM Trans.\ Mathematical Software (2021--Present); ICIAM (2023); Supercomputing (2021); Visual Computer Journal (2021); Quantum Information Processing (2021); JMLR (2019); IEEE SoutheastCon (2018--2020)

\goodbreak

\medskip

\textbf{\large Minisymposium Organizer}

\nopagebreak \medskip \nopagebreak

Multiobjective Optimization Software track in SIAM Conference on Optimization (2021); Geometric Methods for Machine Learning track in SIAM Conference on Computational Science and Engineering (2021)

\goodbreak

\end{document}
